\section*{Referências}

\noindent ALVES, Jéssica Correia Santos. Análise de Controle Ótimo de um Modelo de Crescimento Tumoral com Quimioterapia e Dieta Cetogênica. Dissertação (Mestrado) – Universidade Federal de Juiz de Fora, Instituto de Ciências Exatas. Programa de Pós-Graduação em Matemática, 2020.
\vspace{8pt}

\noindent BASSANEZI, R. C. Equações Diferenciais Ordinárias - Um curso introdutório. São Paulo: UFABC, 2002.
\vspace{8pt}

\noindent BASSANEZI, R.C. Ensino-aprendizagem com Modelagem Matemática: uma nova estratégia. São Paulo: Editora Contexto, 2002.
\vspace{8pt}

\noindent BOYCE, W. E.; DIPRIMA, R. C. Equações Diferenciais Elementares e Problemas de Valores de Contorno. Tradução de Horacio Macedo. 6. ed. rev. Rio de Janeiro: LTC, 1999.
\vspace{8pt}

\noindent DOMINGUES, J. S. Modelo matemático e computacional do surgimento da angiogênese em tumores e sua conexão com as células-tronco. In. Dissertação de Mestrado - Belo Horizonte - MG. CEFET, 2010.
\vspace{8pt}

\noindent DOMINGUES, J. S. Análise do modelo de gompertz no crescimento de tumores sólidos e inserção de um fator de tratamento. Biomatemática IMECC - Unicamp, Campinas, n. 21, p. 103–112, 2011.
\vspace{8pt}

\noindent MELO, Igor Raphael Silva de, SILVA, Noemita Rodrigues da, HUANCA, Roger Ruben Huaman. A dinâmica de crescimento de um tumor: uma proposta para o Ensino de Equações Diferenciais Ordinárias na Licenciatura em Matemática. Tecnologia, investigação, sustentabilidade e os desafios do século XXI... Campina Grande: Realize Editora, 2020. p. 13-32. Disponível em: <https://editorarealize.com.br/artigo/visualizar/64911>. Acesso em: 10/09/2022.
\vspace{8pt}
