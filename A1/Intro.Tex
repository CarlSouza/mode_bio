\section{Introdução}

A matemática é uma grande associada no estudo e na investigação do controle de doenças. No século XVIII uma das suas primeiras incentivos, com o médico, físico e matemático Daniel Bernoulli (1700--1782) para a erradicação da varíola na Europa (SANTOS, 2016). Pelo meio de modelos matemáticos que buscam descrever o comportamento, a evolução e a disseminação de determinadas doenças podemos estudar e fazer experimentos computacionais sem afetar diretamente animais e pessoas.

De acordo com o Instituto Nacional de Câncer ``José de Alencar Gomes da Silva'' - INCA (2020) ``[...] o câncer é o principal problema de saúde pública no mundo e já está entre as quatro principais causas de morte prematura (antes dos 70 anos de idade) na maioria dos países [...]''. As formas de diagnóstico e tratamento são inúmeras, como também as pesquisas para melhorar--los e garantir uma sobrevida aos pacientes.

Ainda segundo Instituto Nacional de Câncer – INCA (2018), o tumor é provocado por um desequilíbrio no sistema de divisão celular, ou seja, o crescimento exagerado de células anormais que acabam acarretando um aumento de tamanho em algum tecido do corpo, e assim atingindo algum órgão.

O presente trabalho apresenta alguns resultados de pesquisas elaboradas sobre modelos matemáticos aplicados aos estudos da dinâmica de crescimento de tumores sólidos, uma proposta para o ensino e aprendizagem de Equações Diferenciais Ordinárias.

Dessa forma, a pesquisa favorece as discussões que apontam que o ensino das Equações
Diferenciais Ordinárias em cursos do Ensino Superior está passando por importantes
transformações nesses últimos anos, no qual, autores como Almeida e Borssoi (2004), Dullius (2009), Habre (2003), Javaroni (2007), Rasmussen (2001) e Stephan e Rasmussen (2002) já vem apresentando estratégias que apontam dinamizar o ensino dessa disciplina além da forma tradicional, ou seja, ir além da resolução analítica dessas equações e assim passando para um olhar mais crítico-exploratório que impulsione a interpretação desses resultados por meio de recursos didáticos que possibilitem uma melhor compreensão conceitual, principalmente quando são trabalhado modelos matemáticos no campo da Matemática Aplicada (FROTA; NASSER, 2009).

O processo de Crescimento Populacional, por exemplo, é uma aplicação que frequentemente é abordada em cursos de Equações Diferenciais Ordinárias -- EDO, nos Cursos de Graduação em Licenciatura em Matemática. Logo no inicio do curso é apresentado definições e os diferentes tipos de EDO (1º Ordem, 2º Ordem  etc) modelos necessários que descrevem o comportamento variante de uma determinada população são apresentados de modo a mostrar a operacionalização da Matemática no desenvolvimento conceitual das ciências físicas, biológicas, químicas, sociais e econômicas, contudo, na disciplina, o foco ainda se detém as resoluções analíticas dessas equações (BOYCE; DIPRIMA, 2002).

Conforme com Zill e Cullen (2001), a montagem de um modelo pode ser um processo simples, complexo ou até mesmo impossível, levando em conta que não basta apenas construir, como também, resolve-lo. Quanto mais aproximado for a descrição do modelo da vida real mais difícil será sua resolução. 

Assim, nota-se que a dinâmica populacional já vem sendo estudada há vários anos por pesquisadores, não só da área de Matemática Pura e Aplicada, mas também por estudiosos de áreas afins na qual a Matemática pode ser aplicada e consequentemente proporciona bons resultados no mundo cientifico (BASSANEZI, 2002).

Portanto nesse artigo, estudamos a modelagem matemática do câncer juntamente com alguns de seus tratamentos, propondo um modelo matemático de crescimento tumoral com a interação das populações de células cancerosas, imunológicas e normais, a ação da quimioterapia e da dieta cetogênica, atuando como tratamentos principal e adjuvante (não farmacológico), respectivamente.

A partir da ação dos tratamentos estudados, temos como principal objetivo otimiza--los, de modo que seja possível reduzir ou eliminar a quantidade de células tumorais e minimizar os possíveis efeitos colaterais provenientes dos tratamentos. 
