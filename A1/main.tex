\documentclass[
	% -- opções da classe memoir --
	article,			% indica que é um artigo acadêmico
	11pt,				% tamanho da fonte
	oneside,			% para impressão apenas no verso. Oposto a twoside
	a4paper,			% tamanho do papel.
	% -- opções da classe abntex2 --
	%chapter=TITLE,		% títulos de capítulos convertidos em letras maiúsculas
	%section=TITLE,		% títulos de seções convertidos em letras maiúsculas
	%subsection=TITLE,	% títulos de subseções convertidos em letras maiúsculas
	%subsubsection=TITLE % títulos de subsubseções convertidos em letras maiúsculas
	% -- opções do pacote babel --
	english,			% idioma adicional para hifenização
	brazil,				% o último idioma é o principal do documento
	sumario=tradicional
	]{abntex2}
% ---
% PACOTES
% ---

% ---
% Pacotes fundamentais
% ---
\usepackage{lmodern}			% Usa a fonte Latin Modern
\usepackage[T1]{fontenc}		% Selecao de codigos de fonte.
\usepackage[utf8]{inputenc}		% Codificacao do documento (conversão automática dos acentos)
\usepackage{indentfirst}		% Indenta o primeiro parágrafo de cada seção.
\usepackage{nomencl} 			% Lista de simbolos
\usepackage{color}				% Controle das cores
\usepackage{graphicx}			% Inclusão de gráficos
\usepackage{microtype} 			% para melhorias de justificação
\usepackage{amssymb}
\usepackage{amsmath,mathrsfs}
\usepackage{amsthm,amsfonts,dsfont}
\usepackage{enumerate}
\usepackage{amsmath} % for the equation* environment


%%%%%%%%%%%%%%%%%%%%%%%%%%%%%%%%%%
\usepackage{bigstrut}
\usepackage{multirow}
\usepackage{array}
\usepackage{tabularx}
\usepackage{booktabs}

\usepackage{ragged2e}
\renewcommand\tabularxcolumn[1]{>{\Centering}m{#1}}

\usepackage{longtable} % para criar tabela que pode ser quebrado em várias páginas
\usepackage{hhline} % para linhas duplas na tabela



\newcommand{\R}{\mathds{R}}
\newcommand{\N}{\mathds{N}}
\newcommand{\Z}{\mathds{Z}}
\newtheorem{teorema}{Teorema}
\newtheorem{postulado}{Postulado}
\newtheorem{definicao}{Definição}
\newtheorem{exemplo}{Exemplo}
\newtheorem{obs}{Obs.}


% ---
		
% ---
% Pacotes adicionais, usados apenas no âmbito do Modelo Canônico do abnteX2
% ---
\usepackage{lipsum}				% para geração de dummy text
% ---
		
% ---
% Pacotes de citações
% ---
\usepackage[brazilian,hyperpageref]{backref}	 % Paginas com as citações na bibl
\usepackage[alf]{abntex2cite}	% Citações padrão ABNT
% ---

% ---
% Configurações do pacote backref
% Usado sem a opção hyperpageref de backref
\renewcommand{\backrefpagesname}{Citado na(s) página(s):~}
% Texto padrão antes do número das páginas
\renewcommand{\backref}{}
% Define os textos da citação
\renewcommand*{\backrefalt}[4]{
	\ifcase #1 %
		Nenhuma citação no texto.%
	\or
		Citado na página #2.%
	\else
		Citado #1 vezes nas páginas #2.%
	\fi}%


% ---

% ---
% Informações de dados para CAPA e FOLHA DE ROSTO
% ---

%logo da fgv
\begin{figure}
    \centering
    \includegraphics{Figuras/download.png}
    \label{fig:my_label}
\end{figure}

\titulo{Análise de Modelos Matemáticos de Tumores de Câncer}
\tituloestrangeiro{Analysis of Mathematical Models of Cancer Tumors}
\autor{Carlos Souza\thanks{Carlos Souza}
}
\local{Brasil}
\data{}
% ---

% ---
% Configurações de aparência do PDF final

% alterando o aspecto da cor azul
\definecolor{blue}{RGB}{41,5,195}

% informações do PDF
\makeatletter
\hypersetup{
     	%pagebackref=true,
		pdftitle={\@title},
		pdfauthor={\@author},
    	pdfsubject={Modelo de artigo científico com abnTeX2},
	    pdfcreator={LaTeX with abnTeX2},
		pdfkeywords={abnt}{latex}{abntex}{abntex2}{atigo científico},
		colorlinks=true,       		% false: boxed links; true: colored links
    	linkcolor=blue,          	% color of internal links
    	citecolor=blue,        		% color of links to bibliography
    	filecolor=magenta,      		% color of file links
		urlcolor=blue,
		bookmarksdepth=4
}
\makeatother
% ---

% ---
% compila o indice
% ---
\makeindex
% ---

% ---
% Altera as margens padrões
% ---
\setlrmarginsandblock{3cm}{3cm}{*}
\setulmarginsandblock{3cm}{3cm}{*}
\checkandfixthelayout
% ---

% ---
% Espaçamentos entre linhas e parágrafos
% ---

% O tamanho do parágrafo é dado por:
\setlength{\parindent}{1.3cm}

% Controle do espaçamento entre um parágrafo e outro:
\setlength{\parskip}{0.2cm}  % tente também \onelineskip

% Espaçamento simples
\SingleSpacing

% ----
% Início do documento
% ----
\begin{document}

% Retira espaço extra obsoleto entre as frases.
\frenchspacing

% ----------------------------------------------------------
% ELEMENTOS PRÉ-TEXTUAIS
% ----------------------------------------------------------

%---
%
% Se desejar escrever o artigo em duas colunas, descomente a linha abaixo
% e a linha com o texto ``FIM DE ARTIGO EM DUAS COLUNAS''.
%\twocolumn[    		% INICIO DE ARTIGO EM DUAS COLUNAS
%
%---
% página de titulo
\maketitle

\newpage
% resumo em português
\begin{resumoumacoluna}
 %%%%%%%%%%%%%%%%%%%5
 
O câncer é um conjunto de doenças causadas pelo crescimento de células danificadas de formar desordenado , que por sua vez invadem tecidos e orgãos  gerando uma massa chamada de tumor. Por sua vez, ela atua desviando o tráfego de sangue para poder se nutrir, como consequência  prejudica as funções vitais do corpo humano , podendo levá-lo a morte.\\
O crescimento celular pode ser facilmente enquadrado ao de dinâmica  de crescimento populacional. Com o Auxílo da matemática é  possível criar modelos matemáticos por meio de Equações diferencias,
que descrevem esse crescimento , sua reação perante ao tratamento de drogas e outros tratamentos ,até mesmo levando a possibildiade de cirurgias .
\\
Nesse trabalho, estudaremos o modelo de\textbf{Gompertz} ,tanto para o crescimento quando para o tratamento do câncer. Mas antes abordaremos outros modelos matemáticos , desde o mais simples como o de \textbf{Malthus}, como os outros modelos foram surgindo a partir dele ,até chegarmos aos modelos mais recentes, afim de comprender melhor essa dinâmica ,e quais aspectos são relevantes na hora de avaliar e tomar resultados 

 \vspace{\onelineskip}

 \noindent
 \textbf{Palavras-chave}: EDOs,Models , Câncer
 
\end{resumoumacoluna}
% resumo em inglês
\renewcommand{\resumoname}{Abstract}
\begin{resumoumacoluna}
 \begin{otherlanguage*}{english}
 The Cancer is a number of diseases caused by the growth of damaged cells which invade tissues and organs generating a mass called tumor. This kind of tumor acts by diverting blood traffic in order to be nourished, as a consequence, it harms the vital functions of the human body and it can even lead to death.
Cell growth can easily be within the population growth dynamics. With the help of mathematics it is possible to create mathematical models through differential equations that describe this growth, their reaction to drug and other treatments, even leading to the possibility of surgery.

In this essay, we will study the Gompertz model for both growth and cancer treatment. But beforet that, we will approach other mathematical models, from the simplest ones, such as \textbf{Malthus}, as the other models emerged from it, until we reach the most recent models in order to better understand this dynamic and identify which aspects are relevant to evaluate and take results



   \vspace{\onelineskip}

   \noindent
   \textbf{Keywords}: EDOs,Models , Câncer
 \end{otherlanguage*}
\end{resumoumacoluna}


%]  				% FIM DE ARTIGO EM DUAS COLUNAS
% ---

% ----------------------------------------------------------
% ELEMENTOS TEXTUAIS
% ----------------------------------------------------------
\textual



\newpage
% ----------------------------------------------------------
% 1. Cap. Introdução
\section{Introdução}

A matemática é uma grande associada no estudo e na investigação do controle de doenças. No século XVIII uma das suas primeiras incentivos, com o médico, físico e matemático Daniel Bernoulli (1700--1782) para a erradicação da varíola na Europa (SANTOS, 2016). Pelo meio de modelos matemáticos que buscam descrever o comportamento, a evolução e a disseminação de determinadas doenças podemos estudar e fazer experimentos computacionais sem afetar diretamente animais e pessoas.

De acordo com o Instituto Nacional de Câncer ``José de Alencar Gomes da Silva'' - INCA (2020) ``[...] o câncer é o principal problema de saúde pública no mundo e já está entre as quatro principais causas de morte prematura (antes dos 70 anos de idade) na maioria dos países [...]''. As formas de diagnóstico e tratamento são inúmeras, como também as pesquisas para melhorar--los e garantir uma sobrevida aos pacientes.

Ainda segundo Instituto Nacional de Câncer – INCA (2018), o tumor é provocado por um desequilíbrio no sistema de divisão celular, ou seja, o crescimento exagerado de células anormais que acabam acarretando um aumento de tamanho em algum tecido do corpo, e assim atingindo algum órgão.

O presente trabalho apresenta alguns resultados de pesquisas elaboradas sobre modelos matemáticos aplicados aos estudos da dinâmica de crescimento de tumores sólidos, uma proposta para o ensino e aprendizagem de Equações Diferenciais Ordinárias.

Dessa forma, a pesquisa favorece as discussões que apontam que o ensino das Equações
Diferenciais Ordinárias em cursos do Ensino Superior está passando por importantes
transformações nesses últimos anos, no qual, autores como Almeida e Borssoi (2004), Dullius (2009), Habre (2003), Javaroni (2007), Rasmussen (2001) e Stephan e Rasmussen (2002) já vem apresentando estratégias que apontam dinamizar o ensino dessa disciplina além da forma tradicional, ou seja, ir além da resolução analítica dessas equações e assim passando para um olhar mais crítico-exploratório que impulsione a interpretação desses resultados por meio de recursos didáticos que possibilitem uma melhor compreensão conceitual, principalmente quando são trabalhado modelos matemáticos no campo da Matemática Aplicada (FROTA; NASSER, 2009).

O processo de Crescimento Populacional, por exemplo, é uma aplicação que frequentemente é abordada em cursos de Equações Diferenciais Ordinárias -- EDO, nos Cursos de Graduação em Licenciatura em Matemática. Logo no inicio do curso é apresentado definições e os diferentes tipos de EDO (1º Ordem, 2º Ordem  etc) modelos necessários que descrevem o comportamento variante de uma determinada população são apresentados de modo a mostrar a operacionalização da Matemática no desenvolvimento conceitual das ciências físicas, biológicas, químicas, sociais e econômicas, contudo, na disciplina, o foco ainda se detém as resoluções analíticas dessas equações (BOYCE; DIPRIMA, 2002).

Conforme com Zill e Cullen (2001), a montagem de um modelo pode ser um processo simples, complexo ou até mesmo impossível, levando em conta que não basta apenas construir, como também, resolve-lo. Quanto mais aproximado for a descrição do modelo da vida real mais difícil será sua resolução. 

Assim, nota-se que a dinâmica populacional já vem sendo estudada há vários anos por pesquisadores, não só da área de Matemática Pura e Aplicada, mas também por estudiosos de áreas afins na qual a Matemática pode ser aplicada e consequentemente proporciona bons resultados no mundo cientifico (BASSANEZI, 2002).

Portanto nesse artigo, estudamos a modelagem matemática do câncer juntamente com alguns de seus tratamentos, propondo um modelo matemático de crescimento tumoral com a interação das populações de células cancerosas, imunológicas e normais, a ação da quimioterapia e da dieta cetogênica, atuando como tratamentos principal e adjuvante (não farmacológico), respectivamente.

A partir da ação dos tratamentos estudados, temos como principal objetivo otimiza--los, de modo que seja possível reduzir ou eliminar a quantidade de células tumorais e minimizar os possíveis efeitos colaterais provenientes dos tratamentos. 


% ----------------------------------------------------------
% 2. Cap. Metodologia
\newpage
\input{2.Aspectos_Biológicos.tex}

% ----------------------------------------------------------
% 3. Cap. Aspectos Biológicos
\newpage
\input{3.Metodologia.tex}

% ----------------------------------------------------------
% 4. Cap. Considerações Finais
\newpage
\input{4.ConsideraçõsFinais}

% ----------------------------------------------------------
% 5. Cap. Referências
%\input{5.Referências}
\newpage
\input{5.Referências.tex}

\end{document} 
