\documentclass[
	% -- opções da classe memoir --
	article,			% indica que é um artigo acadêmico
	11pt,				% tamanho da fonte
	oneside,			% para impressão apenas no verso. Oposto a twoside
	a4paper,			% tamanho do papel.
	% -- opções da classe abntex2 --
	%chapter=TITLE,		% títulos de capítulos convertidos em letras maiúsculas
	%section=TITLE,		% títulos de seções convertidos em letras maiúsculas
	%subsection=TITLE,	% títulos de subseções convertidos em letras maiúsculas
	%subsubsection=TITLE % títulos de subsubseções convertidos em letras maiúsculas
	% -- opções do pacote babel --
	english,			% idioma adicional para hifenização
	brazil,				% o último idioma é o principal do documento
	sumario=tradicional
	]{abntex2}
% ---
% PACOTES
% ---

% ---
% Pacotes fundamentais
% ---
\usepackage{lmodern}			% Usa a fonte Latin Modern
\usepackage[T1]{fontenc}		% Selecao de codigos de fonte.
\usepackage[utf8]{inputenc}		% Codificacao do documento (conversão automática dos acentos)
\usepackage{indentfirst}		% Indenta o primeiro parágrafo de cada seção.
\usepackage{nomencl} 			% Lista de simbolos
\usepackage{color}				% Controle das cores
\usepackage{graphicx}			% Inclusão de gráficos
\usepackage{microtype} 			% para melhorias de justificação
\usepackage{amssymb}
\usepackage{amsmath,mathrsfs}
\usepackage{amsthm,amsfonts,dsfont}
\usepackage{enumerate}
\usepackage{amsmath} % for the equation* environment


%%%%%%%%%%%%%%%%%%%%%%%%%%%%%%%%%%
\usepackage{bigstrut}
\usepackage{multirow}
\usepackage{array}
\usepackage{tabularx}
\usepackage{booktabs}

\usepackage{ragged2e}
\renewcommand\tabularxcolumn[1]{>{\Centering}m{#1}}

\usepackage{longtable} % para criar tabela que pode ser quebrado em várias páginas
\usepackage{hhline} % para linhas duplas na tabela



\newcommand{\R}{\mathds{R}}
\newcommand{\N}{\mathds{N}}
\newcommand{\Z}{\mathds{Z}}
\newtheorem{teorema}{Teorema}
\newtheorem{postulado}{Postulado}
\newtheorem{definicao}{Definição}
\newtheorem{exemplo}{Exemplo}
\newtheorem{obs}{Obs.}


% ---
		
% ---
% Pacotes adicionais, usados apenas no âmbito do Modelo Canônico do abnteX2
% ---
\usepackage{lipsum}				% para geração de dummy text
% ---
		
% ---
% Pacotes de citações
% ---
\usepackage[brazilian,hyperpageref]{backref}	 % Paginas com as citações na bibl
\usepackage[alf]{abntex2cite}	% Citações padrão ABNT
% ---

% ---
% Configurações do pacote backref
% Usado sem a opção hyperpageref de backref
\renewcommand{\backrefpagesname}{Citado na(s) página(s):~}
% Texto padrão antes do número das páginas
\renewcommand{\backref}{}
% Define os textos da citação
\renewcommand*{\backrefalt}[4]{
	\ifcase #1 %
		Nenhuma citação no texto.%
	\or
		Citado na página #2.%
	\else
		Citado #1 vezes nas páginas #2.%
	\fi}%


% ---

% ---
% Informações de dados para CAPA e FOLHA DE ROSTO
% ---

\begin{figure}
    \centering
    \includegraphics{Figuras/download.png}
    \label{fig:my_label}
\end{figure}
\titulo{Análise de Modelos Matemáticos de Tumores de Câncer}
\tituloestrangeiro{Analysis of Mathematical Models of Cancer Tumors}
\autor{Carlos Souza\thanks{DADOS DO AUTOR}
}
\local{Brasil}
\data{}
% ---

% ---
% Configurações de aparência do PDF final

% alterando o aspecto da cor azul
\definecolor{blue}{RGB}{41,5,195}

% informações do PDF
\makeatletter
\hypersetup{
     	%pagebackref=true,
		pdftitle={\@title},
		pdfauthor={\@author},
    	pdfsubject={Modelo de artigo científico com abnTeX2},
	    pdfcreator={LaTeX with abnTeX2},
		pdfkeywords={abnt}{latex}{abntex}{abntex2}{atigo científico},
		colorlinks=true,       		% false: boxed links; true: colored links
    	linkcolor=blue,          	% color of internal links
    	citecolor=blue,        		% color of links to bibliography
    	filecolor=magenta,      		% color of file links
		urlcolor=blue,
		bookmarksdepth=4
}
\makeatother
% ---

% ---
% compila o indice
% ---
\makeindex
% ---

% ---
% Altera as margens padrões
% ---
\setlrmarginsandblock{3cm}{3cm}{*}
\setulmarginsandblock{3cm}{3cm}{*}
\checkandfixthelayout
% ---

% ---
% Espaçamentos entre linhas e parágrafos
% ---

% O tamanho do parágrafo é dado por:
\setlength{\parindent}{1.3cm}

% Controle do espaçamento entre um parágrafo e outro:
\setlength{\parskip}{0.2cm}  % tente também \onelineskip

% Espaçamento simples
\SingleSpacing

% ----
% Início do documento
% ----
\begin{document}

% Retira espaço extra obsoleto entre as frases.
\frenchspacing

% ----------------------------------------------------------
% ELEMENTOS PRÉ-TEXTUAIS
% ----------------------------------------------------------

%---
%
% Se desejar escrever o artigo em duas colunas, descomente a linha abaixo
% e a linha com o texto ``FIM DE ARTIGO EM DUAS COLUNAS''.
%\twocolumn[    		% INICIO DE ARTIGO EM DUAS COLUNAS
%
%---
% página de titulo
\maketitle

% resumo em português
\begin{resumoumacoluna}
Neste trabalho, %%%%%%%%%%%%%%%%%%%5

 \vspace{\onelineskip}

 \noindent
 \textbf{Palavras-chave}: 
\end{resumoumacoluna}
% resumo em inglês
\renewcommand{\resumoname}{Abstract}
\begin{resumoumacoluna}
 \begin{otherlanguage*}{english}






   \vspace{\onelineskip}

   \noindent
   \textbf{Keywords}: 
 \end{otherlanguage*}
\end{resumoumacoluna}


%]  				% FIM DE ARTIGO EM DUAS COLUNAS
% ---

% ----------------------------------------------------------
% ELEMENTOS TEXTUAIS
% ----------------------------------------------------------
\textual



\newpage
% ----------------------------------------------------------
% 1. Cap. Introdução
\section{Introdução}

A matemática é uma grande associada no estudo e na investigação do controle de doenças. No século XVIII uma das suas primeiras incentivos, com o médico, físico e matemático Daniel Bernoulli (1700--1782) para a erradicação da varíola na Europa (SANTOS, 2016). Pelo meio de modelos matemáticos que buscam descrever o comportamento, a evolução e a disseminação de determinadas doenças podemos estudar e fazer experimentos computacionais sem afetar diretamente animais e pessoas.

De acordo com o Instituto Nacional de Câncer ``José de Alencar Gomes da Silva'' - INCA (2020) ``[...] o câncer é o principal problema de saúde pública no mundo e já está entre as quatro principais causas de morte prematura (antes dos 70 anos de idade) na maioria dos países [...]''. As formas de diagnóstico e tratamento são inúmeras, como também as pesquisas para melhorar--los e garantir uma sobrevida aos pacientes.

Ainda segundo Instituto Nacional de Câncer – INCA (2018), o tumor é provocado por um desequilíbrio no sistema de divisão celular, ou seja, o crescimento exagerado de células anormais que acabam acarretando um aumento de tamanho em algum tecido do corpo, e assim atingindo algum órgão.

O presente trabalho apresenta alguns resultados de pesquisas elaboradas sobre modelos matemáticos aplicados aos estudos da dinâmica de crescimento de tumores sólidos, uma proposta para o ensino e aprendizagem de Equações Diferenciais Ordinárias.

Dessa forma, a pesquisa favorece as discussões que apontam que o ensino das Equações
Diferenciais Ordinárias em cursos do Ensino Superior está passando por importantes
transformações nesses últimos anos, no qual, autores como Almeida e Borssoi (2004), Dullius (2009), Habre (2003), Javaroni (2007), Rasmussen (2001) e Stephan e Rasmussen (2002) já vem apresentando estratégias que apontam dinamizar o ensino dessa disciplina além da forma tradicional, ou seja, ir além da resolução analítica dessas equações e assim passando para um olhar mais crítico-exploratório que impulsione a interpretação desses resultados por meio de recursos didáticos que possibilitem uma melhor compreensão conceitual, principalmente quando são trabalhado modelos matemáticos no campo da Matemática Aplicada (FROTA; NASSER, 2009).

O processo de Crescimento Populacional, por exemplo, é uma aplicação que frequentemente é abordada em cursos de Equações Diferenciais Ordinárias -- EDO, nos Cursos de Graduação em Licenciatura em Matemática. Logo no inicio do curso é apresentado definições e os diferentes tipos de EDO (1º Ordem, 2º Ordem  etc) modelos necessários que descrevem o comportamento variante de uma determinada população são apresentados de modo a mostrar a operacionalização da Matemática no desenvolvimento conceitual das ciências físicas, biológicas, químicas, sociais e econômicas, contudo, na disciplina, o foco ainda se detém as resoluções analíticas dessas equações (BOYCE; DIPRIMA, 2002).

Conforme com Zill e Cullen (2001), a montagem de um modelo pode ser um processo simples, complexo ou até mesmo impossível, levando em conta que não basta apenas construir, como também, resolve-lo. Quanto mais aproximado for a descrição do modelo da vida real mais difícil será sua resolução. 

Assim, nota-se que a dinâmica populacional já vem sendo estudada há vários anos por pesquisadores, não só da área de Matemática Pura e Aplicada, mas também por estudiosos de áreas afins na qual a Matemática pode ser aplicada e consequentemente proporciona bons resultados no mundo cientifico (BASSANEZI, 2002).

Portanto nesse artigo, estudamos a modelagem matemática do câncer juntamente com alguns de seus tratamentos, propondo um modelo matemático de crescimento tumoral com a interação das populações de células cancerosas, imunológicas e normais, a ação da quimioterapia e da dieta cetogênica, atuando como tratamentos principal e adjuvante (não farmacológico), respectivamente.

A partir da ação dos tratamentos estudados, temos como principal objetivo otimiza--los, de modo que seja possível reduzir ou eliminar a quantidade de células tumorais e minimizar os possíveis efeitos colaterais provenientes dos tratamentos. 


% ----------------------------------------------------------
% 2. Cap. Metodologia
\input{2.Metodologia}

% ----------------------------------------------------------
% 3. Cap. Aspectos Biológicos
\input{3.Aspectos_Biológicos}

% ----------------------------------------------------------
% 4. Cap. Considerações Finais
%\input{4.ConsideraçõsFinais}

% ----------------------------------------------------------
% 4. Cap. Referências
%\input{5.Referências}


\end{document} 
